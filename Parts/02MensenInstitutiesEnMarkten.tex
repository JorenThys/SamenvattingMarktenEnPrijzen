\section{Mensen, Instituties en Markten}

Maatschappelijke organisatie speelt grote rol in het wat/hoe/voor wie producren. Iedereen is slechts klein deeltje v/h ``raderwerk'' en toch loopt alles nog goed.

Eerst kijken we naar het gedrag van een individu en aggregeren dit dan tot een model dat dit co\"{o}rdinatieprobleem kan beantwoorden.

\subsection{Individuele Gedragsmodel}
Bij het kijken naar het gedrag v/e individu vertrekken we weer vanuit het axioma van rationeel gedrag. De consument is een individu met een nutsfunctie $U(q_1,q_2)$. Deze nutsfunctie wil hij maximaliseren gegeven een bepaalde beperking van zijn budget $y =  p_1 q_1 + p_2 q_2$. We willen onder die restrictie zo'n hoog mogelijk nut/voldoening hebben.

De producent aan de andere kant wil zijn totale kosten ($TK$) zo laag mogelijk houden, waarbij $TK = p_A q_A + p_K q_K$ waarbij $A$ voor Arbeid staat en $K$ voor kapitaal. De producent gaat dus op zoek naar een optimale mix van arbeid en kapitaal (productiefactoren) die een zo hoog mogelijke productie opleveren.

Gegeven het individuele gedragsmodel kunnen we verband leggen met maatschappelijke uitkomsten via het concept van \textbf{evenwicht}. Een econ. agent is in evenwicht als er geen interne krachten zijn waardoor hij zijn gedrag zou willen veranderen $\rightarrow$ de agent heeft zijn optimum bereikt.

\textbf{Globaal evenwicht} is wanneer alle econ. agenten hun optimum bereikt hebben.

Maar hoe het evenwicht van een individu vinden? Ofwel via individuele beslissingen of sociale interacties.

\subsubsection{Individuele Beslissingen}
Geen onderlinge afhankelijkheid tussen de econ. agenten, geen rekening houden met de evt. reacties van anderen.

\subsubsection{Sociale Interacties}
Wel reacties van andere econ. agenten op elkaars gedrag. Individueel gedrag nu moeilijker te achterhalen $\rightarrow$ \textbf{speltheorie}.
Individueel optimum valt echter niet altijd samen met maatschappelijk optimum. Soms zijn hiervoor soms informele sociale instituties (bindende afspraken \& sociale normen) nodig, of tussenkomen v/d overheid, indien sociale normen geen \textbf{Nash-evenwicht} bekomen.

Er zijn een aantal systemen die voorkomen voor arbeidsverdeling en co\"{o}rdinatie.
\begin{description}
	\item[Traditionele Systemen] Wat en hoe gedicteerd door \textbf{traditionele regels/sociale normen}. Niet geschikt voor dynamische omgevingen.
    \item[Bevelsystemen] Centrale instantie regelt wie wat/hoe/hoeveel/voor wie produceert. Nood aan het verzamelen van veel informatie en er ontstaat een moeilijk te overkomen \textbf{incentiefprobleem}.
    \item[Marktsystemen] Econ. agenten beslissen zelf over alles, met als doen zelf zo goed mogelijk af te zijn. Alhoewel handel niet  altijd spontaan tot stand komt, er is nood aan duidelijk gedefini\"{e}erd eigendomsrecht, zal dit wel altijd voordeel leveren aan de betrokken partijen. \textbf{Prijzen} zijn belangrijk bij marktsystemen. Ze geven signalen over voorkeuren v/d consumenten en kosten v/d producenten. Daarnaast zijn veranderingen in prijs ook een incentief voor econ. agenten om hun gedrag aan te passen. Er speelt wel een \textbf{informatieprobleem} . In sommige gevallen, zoals tabel~\ref{tab:tweedeHands} gaan uiteindelijk de aanbieders van goede wagens zich terugtrekken. Dit omdat de verwachte waarde van de aankoop EUR 2600 is(gewogen gemiddelde als je weet dat 2/3 auto's slecht is). Dit is onder de vraagprijs voor goede auto's, dus deze gaan verdwijnen. Dit herhaalt zich tot de markt uitdooft.
\end{description}
\begin{table}[h]
	\centering
    \begin{tabular}{ | c |  c |  c | }
		\hline
		& Door verkoper gevraagde prijs  &  Bereidheid tot betalen \\
		\hline
		 Goede auto & EUR 2700 & EUR 3000 \\
		\hline
		Slechte auto & EUR 2100 & EUR 2400\\
		\hline
	\end{tabular}
	\caption{Voorbeeld informatieprobleem op tweedehandsmarkt}
    \label{tab:tweedeHands}
\end{table}
